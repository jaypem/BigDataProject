
In dieser Arbeit wurde eine Sentiment-Analyse von Twitterdaten mit Hilfe von Spark durchgef\"uhrt. Da dabei Python genutzt wurde, sind bei der Textanalyse Schwierigkeiten mit der Kodierung aufgetreten, die eine Bereinigung der Twitterdaten n\"otig machen. \\
Die Sentiment-Analyse mit Hilfe von \textit{TextBlob} kann ausschliesslich auf englischsprachigen Texten durchgef\"uhrt werden. Auch wenn eine Erkennung der Sprache angeboten und eine anschlie{\ss}ende \"Ubersetzung mit Hilfe von \textit{Google Translate} m\"oglich ist, ist dieses Verfahren sehr zeitaufwendig. Aus diesem Grund ist die Analyse in dieser Arbeit auf englischsprachige Tweets beschr\"ankt. Trotzdem k\"onnte eine detailliertere Auswertung des kompletten Datensatzes interessant sein.\\
Um bessere R\"uckschl\"usse auf einen m\"oglichen Zusammenhang zwischen Bitcoin-Kurs und Twitterdaten zu erlauben, ist ein konsistenter Datensatz notwendig, der keine zahlreichen zeitlichen L\"ucken aufweist. Zus\"atzlich k\"onnten neben Twitter auch andere Quellen genutzt werden, um die \"offentliche Meinung zum Thema Bitcoin auszuwerten. Ein Beispiel daf\"ur w\"are das Soziale Netzwerk \textit{Facebook}. \\
In Abbildung \ref{fig:result} wird deutlich, dass der Anteil an neutralen Nachrichten besonders gro{\ss} ist. Um ein genaueres Bild dieser Meldungen zu erhalten sollte der gew\"ahlte Klassifikator genauer betrachtet werden. Eine M\"oglichkeit ist eine detallierte Untersuchung des Verhaltens des Klassifikators auf verschiedene Texte, um gegebenenfalls die gew\"ahlten Schwellwerte anzupassen. Eine andere Option ist das Trainieren eines eigenen Klassifikators, der auf die Problemstellung angepasst ist. \\
Schlussendlich ist eine analoge Analyse nicht nur f\"ur den Begriff \textit{Bitcoin}, sondern auch f\"ur andere Kryptow\"ahrungen m\"oglich.
